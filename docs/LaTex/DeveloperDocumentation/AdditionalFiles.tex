\section{\en{Files outside the main app} \de{Dateien außerhalb von App}}
\subsection{./projectmanagement}
\en{This folder  contains the description and images for the release in the Google Play Store.} \de{Enthält die Beschreibung und die Screenshots für den Google Play Store.}

\subsection{./docs}
\subsubsection{LaTex/DeveloperDocumentation}
\en{Contains the LaTex files for this documentation. The UML diagrams were generated using an Android Studio plugin.} \de{Enthält die LaTex Dateien für diese Dokumentation. Die UML Diagramme wurden mithilfe eines Android Studio Plugins erzeugt.}

\subsubsection{fastlane-screenshots}
\en{Folder for fastlane to store it's automatically generated screenshots, which can be used in documentation or otherwise. These screenshots are generated in different themes, states, and languages.} \de{Hier speichert Fastlane seine automatisch generierten Screenshots, welche in der Dokumentation oder anderweitig verwendet werden können. Alle Screenshots werden in verschieden Farbschemen, Zuständen und Sprachen generiert.}

\subsection{./android-app}
\begin{itemize}
	\item \textbf{Gemfile} \en{Contains the gems needed for fastlane.} \de{Enthält die Gems für Fastlane.}
	\item \textbf{Gemfile.lock} \en{Locks all the dependency versions.} \de{Speichert die Version für alle nötigen Abhängigkeiten.}
\end{itemize}

\subsubsection{.run}
Contains all run configurations for Android Studio.
\begin{itemize}
	\item \textbf{Fastlane Screenshots} \en{A run configurations to generate new screenshots using fastlane. Make sure to follow \ref{sec:installFastlane} first.} \de{Eine Konfiguration um mit Fastlane neue Screenshots zu erzeugen. Zuvor ist \ref{sec:installFastlane} zu folgen.}
\end{itemize}

\subsubsection{fastlane}
\en{Contains the fastlane files to automatically generate screenshots.} \de{Enthält die Fastlane Dateien, um automatisch Screenshots zu erzeugen.}